%%
% 引言或背景
% 引言是论文正文的开端,应包括毕业论文选题的背景、目的和意义;对国内外研究现状和相关领域中已有的研究成果的简要评述;介绍本项研究工作研究设想、研究方法或实验设计、理论依据或实验基础;涉及范围和预期结果等。要求言简意赅,注意不要与摘要雷同或成为摘要的注解。
% modifier: 黄俊杰(huangjj27, 349373001dc@gmail.com)
% update date: 2017-04-15
%%

\chapter{绪论}
%定义,过去的研究和现在的研究,意义,与图像分割的不同,going deeper
\label{cha:introduction}
\section{选题背景与意义}
\label{sec:background}

随着互联网技术的发展,数字化转型过程的加速,人们对信息安全的重视程度不断提高。传统的身份验证方式,如密码、验证码或身份卡等,已经无法满足用户在实际应用中对便捷性、可靠性和安全性的多重需求。
传统的身份验证技术面临诸多问题:多平台的不同密码容易被用户遗忘,钥匙和身份证件等物理令牌存在被盗用风险,可能丢失或遭受物理损坏而失效。

为解决这些问题,满足用户需求,近几十年来,基于生物特征的识别技术逐渐兴起。这种技术通过分析用户的个人生理特征(如面部\cite{turk1991face}、指纹\cite{jain1997line}、虹膜\cite{daugman2009iris}和静脉\cite{qin2017deep})
或行为特征(如步态\cite{wang2003silhouette}和声音\cite{perrachione2011human})来自动识别个人身份,并且以此建立的一些商业化的生物识别系统,广泛应用于移民清关、财务支付和访问控制等领域。

然而,基于人脸、指纹和虹膜等生物特征的识别技术在安全性、便利性方面仍存在问题。
例如,人脸信息容易在用户不知情的情况下被窃取\cite{chingovska2012effectiveness},冒名顶替者可以利用照片、数字视频甚至三维面具\cite{nesli2013spoofing}通过识别系统;
指纹样本也容易从传感器上的残留物中被复制,导致信息泄露;虹膜识别技术虽然准确度高,但对使用场景和设备要求较高,便利性不足。
此外,生物特征泄露导致的用户终身风险也无法忽视,生物特征数据库的安全性值得更多的关注。

相比之下,静脉特征具有独特的优势。这包括:
静脉丛的结构在成年后保持恒定(误差<$0.3\%$/年),且不受表皮污染、磨损的影响(掌纹等皮肤纹识别技术的缺点);
静脉网络位于3-5mm真皮层下,需近红外(700-1000nm)成像,规避了可见光下的信息泄露风险;
血红蛋白氧合状态直接影响血管显影效果,确保静脉信息无法通过离体组织伪造。

由于这些优势,近年来,静脉识别技术得到了越来越多的关注。其中指静脉、手静脉和掌静脉识别技术有了广泛研究和应用。
指静脉识别便捷,但血管结构简单,识别准确性低。相较于指静脉识别,掌静脉系统具有更丰富的特征维度。
掌静脉的平均特征点数量达32$\pm$6个,显著高于指静脉的12$\pm$3个,信息熵提升2.4倍,理论上可将碰撞概率降至$10^{-15}$量级

在\autoref{tab:bioreco}中给出了常见生物特征识别技术的信息,利用EER
\footnote{
    EER(平均错误概率)是一种生物识别安全系统算法,用于预先确定其错误接受率及其错误拒绝率的阈值。定义为错误接受率(FAR)与错误拒绝率(FRR)相等时的阈值点;或者说是ROC曲线中正负样本错分概率相等的点所对应的错分概率值。EER等错误率值越低,生物识别系统的准确度越高
    }
    指标评价以此生物特征进行识别的准确率。

\begin{table}[h] %voc table result
    \centering
    \caption{生物特征识别技术对比}
    \begin{tabular}{*{5}{c}}
        \toprule
        特征类型    & 采集方式      & EER(\%)      & 防伪等级     & 用户接受度      \\
        \midrule
        指纹       & 接触式         & 0.5           & 中           &  7.8              \\
        虹膜       & 非接触式       & \textbf{0.01} & 高           &  6.2             \\
        人脸       & 非接触式       & 1.2           & 低           &  9.1               \\
        静脉       & 非接触式       & \textbf{0.08} & 极高         &  8.5               \\
        \bottomrule
    \end{tabular}
    \label{tab:bioreco}
\end{table}

\section{国内外研究现状和相关工作}
\label{sec:related_work}

静脉作为脉状图案在可见光下很难观察到,只用波长约为850nm的红外光收集成像。
收集的静脉图像可能受到许多因素的影响,例如光散射\cite{yang2014towards}、环境温度\cite{kumar2011human}和用户行为\cite{miura2007extraction}等。
成像中可能有一些难以区分静脉区域和背景区域的低质量部分,这使得高准确识别静脉极具挑战性。为解决识别困难,研究人员提出了许多方法用于掌静脉识别,主要有以下三类:

1)手工方法:通常是基于数学的模型,应用先验知识来克服识别分类困难\cite{nanni2017handcrafted}。包括基于谷值检测的方法\cite{miura2007extraction},基于线状检测的方法\cite{kumar2011human}和基于局部描述符的方法\cite{kang2014contactless}。

2)传统机器学习的方法:以浅层结构处理图像数据\cite{wang2021comparative}。不同的经典机器学习算法从数据中学习静脉分布规律,对新的样本数据做出识别预测。常用算法包括PCA及其变体,LDA,SVM等。此类机器学习模型最多只有一层或两层非线性特征变换

3)基于深度学习的方法:与传统机器学习的浅层结构不同,深度学习强调模型结构的深度和特征学习的重要性,以端到端方式从数据中自动学习特征\cite{wang2021comparative}。近年来,深度神经网络技术已展现出出强大的特征表示能力,在计算机视觉任务中取得了优秀的结果。
受其启发,许多研究人员将深度神经网络带入了静脉分类任务\cite{itqan2016user},静脉分割与识别任务\cite{qin2017deep}。

手工方法根据一定程度的假设提取静脉,实验证明存在相当的局限性。如经典的谷值检测法虽能有效提取静脉血管中心线,但在低对比度图像上的的误检率超20\%。多光谱融合提出双波段成像以提高红外光穿透深度,但设备成本增加3倍且帧率受限。
传统的机器学习的方法(如SVM和LDA)能够在一定程度上学习图像的稳健特征。然而,这些机器学习方法的特征表示能力终究有限(与深度学习方法相比),识别性能仍然不够稳健。

基于深度学习的方法能够通过从大型训练数据集中提取特征信息。在图像领域中,深度学习方法从原始图像中自动学习稳健的特征表示,无需任何事先假设。
尽管如此,深度学习技术在掌静脉识别中的应用仍处于发展阶段。正如相关研究\cite{jia2021performance}的结论中所述,经典卷积神经网络(CNN)在识别准确率低于某些传统方法的一个重要原因就是:深度学习的参数量过大,这需要大量的训练样本来不断优化参数。
而由于存储的限制和生物特征隐私政策,研究者难以获取大量、高分辨的生物特征图像训练样本。同时,实验结果也证实了即使是不完美的合成数据,也可以提高分类器的性能。

为解决数据量不足,模型欠训练的问题,一些研究人员引入了生成模型来实现数据增强
\footnote{
    数据增强(Data augmentation)是一种统计技术,允许从不完整数据中进行最大似然估计,在机器学习中广泛使用。训练模型使用已有数据的几个略微修改的副本即可在学习模型时减少过拟合。
    在图像分类、识别任务中,数据增强已成为一种基础工具,用来丰富训练数据集的多样性,以提升模型的泛化能力和性能。几何变换、颜色空间调整和噪声注入等是数据增强在图像分类中的常用工具。
    }。
例如,利用生成对抗模型(GAN)扩大掌静脉训练样本数量\cite{ou2022gan}。在掌纹识别\cite{wang2018generative}和指静脉识别\cite{zhang2019gan}的研究中也可以看到GAN生成技术的应用。
另一种生成模型扩散模型(Diffusion),因其生成高质量图像的能力,在图像生成(包括医学成像)领域已有广泛应用\cite{zhang2023survey}。研究人员已将扩散模型应用到指静脉数据的增强中\cite{liu2024diffvein},也可以预想其在掌静脉数据增强中的强大潜力。

\section{本文的论文结构与章节安排}

\label{sec:arrangement}

本文共分为六章,各章节内容安排如下:

第一章绪论:简单说明了本文章的选题背景与意义。(关于掌静脉在生物识别中的优势与研究意义,说明大规模高质量掌静脉图片生成的需求,主要用于提高掌静脉的深度学习算法的是识别效果)

第二章图像生成模型:介绍对抗生成模型与扩散模型思想,包括DCGAN框架和DDPM原理。

第三章技术改进:在生成对抗模型中,应用标签平滑化,特征匹配技术;在扩散模型中,创新性地在掌静脉数据增强中使用更合适的余弦加噪方式替换线性加噪方式。

第四章实验结果:数据集介绍,DCGAN图像增强结果,DDPM图像增强结果。

%第五、六章是本文的最后两章,作为空白章节例子。


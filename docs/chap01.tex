%%
% 引言或背景
% 引言是论文正文的开端,应包括毕业论文选题的背景、目的和意义;对国内外研究现状和相关领域中已有的研究成果的简要评述;介绍本项研究工作研究设想、研究方法或实验设计、理论依据或实验基础;涉及范围和预期结果等。要求言简意赅,注意不要与摘要雷同或成为摘要的注解。
% modifier: 黄俊杰(huangjj27, 349373001dc@gmail.com)
% update date: 2017-04-15
%%

\chapter{绪论}
%定义,过去的研究和现在的研究,意义,与图像分割的不同,going deeper
\label{cha:introduction}
\section{选题背景与意义}
\label{sec:background}

随着互联网技术的发展与普及,人们对信息安全重视程度也在不断提高。传统的身份验证方式,如密码、个人识别码或智能卡等,已经无法满足用户在实际应用中对便捷性、可靠性和安全性的多重需求。
常见的身份验证技术存在诸多问题,例如不同平台的密码容易被遗忘,对安全性的要求也使得密码越来越复杂,难以记录。钥匙易于被复制和伪造,身份证件存在被盗用风险且可能丢失或遭受物理损坏。

为解决这些问题,近几十年来,基于生物特征的识别技术逐渐兴起。这种技术通过分析用户的个人生理特征(如面部\cite{turk1991face}、指纹\cite{jain1997line}、虹膜\cite{daugman2009iris}和静脉\cite{qin2017deep})
或行为特征(如步态\cite{wang2003silhouette}和声音\cite{perrachione2011human})来自动识别个人身份,并且以此建立的一些商业化的生物识别系统,广泛应用于移民清关、财务支付和访问控制等领域。

然而,人脸、指纹和虹膜等生物特征识别技术在安全性和便利性方面仍存在一些问题。
例如,人脸信息容易在用户不知情的情况下被窃取\cite{chingovska2012effectiveness},冒名顶替者可以利用照片、数字视频甚至三维面具\cite{nesli2013spoofing}通过识别系统;
指纹样本也容易从传感器上的残留物中被复制,导致信息泄露;虹膜识别技术虽然准确度高,但对使用场景和设备要求较高,便利性不足。

相比之下,静脉特征具有独特的优势。静脉图案隐藏在人体皮肤之下,因此具有以下特点:

1)活体验证:只能从活体中采集静脉图案,确保了验证的有效性和高效性。

2)高安全性和隐私性:静脉图案难以伪造,且在未经用户同意的情况下难以获取。

基于这些优势,静脉识别技术近年来受到了越来越多的关注。其中,指静脉、手静脉和掌静脉识别技术得到了广泛研究和应用。尽管指静脉识别较为便捷,但其血管结构相对简单,复杂性和唯一性不如掌静脉,且较容易被采集,隐私保护性较弱。掌纹识别则因掌纹特征位于体表,容易被复制或盗用,隐私保护性较差,且易受外部条件(如污渍、磨损)影响,精度和安全性较低。因此,掌静脉在高精度识别和大规模应用方面的优势更加突出,受到了更多关注。
\section{国内外研究现状和相关工作}
\label{sec:related_work}

静脉作为脉状图案在可见光下很难观察到,但它们可以用波长约为850nm的红外光收集。
然而,识别静脉仍然具有挑战性的任务,因为收集的静脉图像可能受到许多因素的影响,例如光散射\cite{yang2014towards}、环境温度\cite{kumar2011human}和用户行为\cite{miura2007extraction}。
收集过程会生成一些难以区分静脉区域和背景区域的低质量图像,降低识别准确性。为了解决这个问题,许多传统方法被提出用于掌静脉识别,主要有以下三类:

1)手工方法:通常是基于数学的模型,通过先验知识来克服特定问题\cite{nanni2017handcrafted}。包括基于谷值检测的方法\cite{zhou2011human}\cite{miura2007extraction},基于线状检测的方法\cite{kumar2011human}和基于局部描述符的方法\cite{kang2014contactless}。

2)传统的基于机器学习的方法:大多数传统的机器学习方法以浅层结构处理数据\cite{wang2021comparative}。模型最多只有一层或两层非线性特征变换。与手工方法不同,传统的机器学习是通过算法从数据中学习分布规律,并对新的样本数据做出预测或判断。包括PCA及其变体,LDA,SVM算法等。

3)基于深度学习的方法:与传统的浅层学习不同,深度学习强调模型结构的深度和特征学习的重要性,它可以以端到端的方式自动从数据中学习特征\cite{wang2021comparative}。近年来,CNN和Transformer等深度神经网络已经显示出强大的特征表示能力,并在计算机视觉任务中取得了优秀的结果。受其成功的启发,许多研究人员将它们带入了静脉分类任务\cite{itqan2016user},例如,CNN 已被用于静脉识别 \cite{itqan2016user}、质量评估\cite{qin2017deep}、静脉分割\cite{qin2017deep}等任务。

手工方法根据一定假设,根据分布特征提取静脉。但是,静脉在像素上的表现可能会产生复杂的分布,从而影响识别性能。另外,手工方法也通常会提取一些基于图像处理后的特征,这些特征可能会丢失一定程度的有效信息。因此,手工方法存在局限性,提取的结果可能不完整,有效识别性不足。
传统的机器学习的方法(如SVM和LDA)能够在一定程度上学习图像的稳健特征。然而,这些方法的特征表示能力有限(与深度学习方法相比),识别性能仍然不够稳健。

与前两类方法不同,基于深度学习的方法能够通过从大型训练数据中推断出丰富的信息,从原始图像中自动学习稳健的特征表示,而无需任何事先假设。
然而,基于深度学习的掌静脉识别研究仍然处于发展阶段,正如相关文章\cite{jia2021performance}的结论中所述,一些经典的卷积神经网络(CNN)在数据集上的识别准确率低于某些传统方法。
一个重要原因是,深度学习网络需要大量的训练样本来训练大量的网络参数。
然而,存储的限制和隐私政策对生物特征的保护,使得在实际应用中难以获取大量、优质的训练样本。
为了解决这个问题,一些研究人员引入了生成模型来实现数据增强。例如利用GAN来增强掌静脉训练样本集\cite{ou2022gan},扩大样本量。GAN已被用于手掌静脉识别\cite{ou2022gan}、掌纹识别\cite{wang2018generative}和指静脉识别\cite{zhang2019gan}的训练图像集。
即使是不完美的合成数据也可以提高分类器的性能。
扩散模型作为另一种生成模型,因其生成高质量图像的能力,在图像生成(包括医学成像)中的已经有了广泛应用\cite{zhang2023survey}。该模型通过逐步的扩散过程,捕捉数据的复杂依赖关系。在静脉图像生成中,这种能力有助于更准确地模拟静脉的结构和纹理特征。在一些研究中,已经应用到指静脉数据的增强中\cite{liu2024diffvein},我们可以预想其在掌静脉数据增强中的强大潜力。


\section{本文的论文结构与章节安排}

\label{sec:arrangement}

本文共分为六章,各章节内容安排如下:

第一章绪论。简单说明了本文章的选题背景与意义。(关于掌静脉在生物识别中的优势与研究意义,说明大规模高质量掌静脉图片生成的需求)

第二章图像生成模型(对抗生成模型与扩散模型)

第三章技术改进:对抗生成模型中(标签平滑化,反卷积切换为上采样+卷积,学习率衰减);扩散模型中()

第四章实验结果

第五、六章是本文的最后两章,作为空白章节例子。


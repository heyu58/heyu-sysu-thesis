%%
% 摘要信息
% 本文档中前缀"c-"代表中文版字段, 前缀"e-"代表英文版字段
% 摘要内容应概括地反映出本论文的主要内容,主要说明本论文的研究目的、内容、方法、成果和结论。要突出本论文的创造性成果或新见解,不要与引言相 混淆。语言力求精练、准确,以 300—500 字为宜。
% 在摘要的下方另起一行,注明本文的关键词(3—5 个)。关键词是供检索用的主题词条,应采用能覆盖论文主要内容的通用技术词条(参照相应的技术术语 标准)。按词条的外延层次排列(外延大的排在前面)。摘要与关键词应在同一页。
% modifier: 黄俊杰(huangjj27, 349373001dc@gmail.com)
% update date: 2017-04-15
%%

\cabstract{
    随着互联网技术的发展,公众对身份验证技术在安全性与便捷性方面的要求愈来愈高。传统身份验证技术有不同的缺点,相比之下,新兴的生物识别技术展现出更好的前景。
    本文系统分析了传统身份验证技术的局限性,对比了常见的生物识别技术特点,重点探讨静脉识别技术的最新进展。针对静脉识别技术中常用的方法进行整理分析,对比了深度学习与传统方法的实验效果。
    深度学习技术在静脉识别方面稳健的学习效果,让该技术对公众身份大规模快速精确识别成为可能。接着,由于隐私政策和成本问题,深度学习方法在静脉识别领域面临数据量不足,模型欠学习的问题。
    本文参考相关文献,讨论基于两种生成式模型的数据增强解决方案。

    针对掌静脉图像的特点,本文基于生成对抗模型与扩散模型进行相应的改进,探讨了两种技术的优缺点,对比了改进效果与生成图像质量。
    研究发现,融合扩散模型的深度学习方法在静脉特征表达和跨设备泛化能力方面展现出更显著优势,这为突破生物特征识别的小样本困境提供了新思路。
    
    最后,本文讨论了生成对抗模型与扩散模型技术融合(GAN+Diffusion)的可能。该技术已经在人脸识别领域得到实现,可以做到图像数据的不同形态(角度,光照,年龄)可控生成,将成为数据增强的新范例。
}
% 中文关键词(每个关键词之间用“,”分开,最后一个关键词不打标点符号。)
\ckeywords{静脉识别,数据增强,生成对抗网络,扩散模型}

\eabstract{
    % 英文摘要及关键词内容应与中文摘要及关键词内容相同。中英文摘要及其关键词各置一页内。
    The content of the English abstract is the same as the Chinese abstract, 250-400 content words are appropriate. Start another line below the abstract to indicate English keywords (Keywords 3-5).
}
% 英文文关键词(每个关键词之间用,分开, 最后一个关键词不打标点符号。)
\ekeywords{DCGAN,DDPM}


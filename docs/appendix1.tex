\chapter{证明细节}
\section{生成对抗网络损失函数推导}
记$l$是样本$x$的真实标签,而$y$则是神经网络判别器$D$给出的认为真样本的可能性,$y=D(\mathbf{x})$
从判别器$D$的角度对某个样本i考虑
$$L_i=y_i^{l_i}(1-y_i)^{(1-l_i)}$$
如果$l_i=1$,上式为$L_i=y_i$,显然$y_i$预测为真的概率越大,预测越正确,$L_i$越大
同理若$l_i=0$,上式为$L_i=1-y_i$,$y_i$预测为真的概率越大,预测越错,$L_i$越小
因此考虑将GAN判别器$D$损失问题转变为极大似然估计问题,如何在一批样本$\{x_1,x_2,...,x_m\}$训练后,最大化预测正确率$L$
%无标号的公式
\begin{align*}
        \underset{D}{max}L=\prod_{i=1}^mL_i=\prod_{i=1}^my_i^{l_i}(1-y_i)^{1-l_i}
\end{align*}
做数学变化,引入单调函数$\log$,并做常见的平均处理,问题转换为等价的优化问题
\begin{align*}
    \begin{split}
        \underset{D}{max}\frac{1}{m}\sum_{i=1}^m\Big[\log y_i^{l_i}+\log(1-y_i)^{1-l_i}\Big] \\
        \underset{D}{max}\frac{1}{m}\sum_{i=1}^m\Big[l_i\log D(x_i)+(1-l_i)D(x_i)\Big]
    \end{split}
\end{align*}
由于标签只有在$x\sim p_{data}(x)$时为1,在$z\sim p_{seed}(x)$时为0,因此上式是期望的样本实现。我们将标签为假的样本写成有生成器输出结果$G(z)$
\begin{align*}
    \begin{split}
        \underset{D}{max}\Big[\mathbb{E}_{x\sim p_{data}(x)}\log D(x)+\mathbb{E}_{z\sim p_{seed}(x)}\big(1-D(G(z))\big)\Big]
    \end{split}
\end{align*}
上式只是对判别器$D$建立了优化目标,显然判别器越强,生成器生成的样本就永远会高概率判断为假。因此生成器的目标是最小化判别器的优化目标
\begin{align*}
    \begin{split}
        \underset{G}{min}\underset{D}{max}\Big[\mathbb{E}_{x\sim p_{data}(x)}\log D(x)+\mathbb{E}_{z\sim p_{seed}(x)}\big(1-D(G(z))\big)\Big]
    \end{split}
\end{align*}
假设生成器$G$有网络参数$\mathbf{\phi}$,判别器$D$有网络参数$\mathbf{\theta}$,我们可以从网络参数的角度给出最终的损失函数

\begin{equation}
    \underset{G}{min}\underset{D}{max}\Big[\mathbb{E}_{x\sim p_{data}(x,\phi)}\log D(x)+\mathbb{E}_{z\sim p_{seed}(x)}\big(1-D(G(z,\theta),\phi)\big)\Big]
\end{equation}

\section{扩散模型正向过程推导部分}
参考\cite{luo2022understanding}\cite{weng2021diffusion}
在扩散模型的前向过程中,来自训练集的图像$\mathbf{x}_0$会被添加$T$次噪声,
使得$\mathbf{x}_T$为符合标准正态分布。大多数前向过程采样的正态分布会设置为下式
\begin{align*}
    \mathbf{x}_t\sim\mathcal{N}(\sqrt{1-\beta_t}\mathbf{x}_{t-1},\beta_t\mathbf{I})
\end{align*}
尝试从$\mathbf{x}_t$持续倒推,$\mathbf{x}_t$可以通过一个标准正态分布的样本$\epsilon_{t-1}\sim\mathcal{N}(0,\mathbf{I})$算出来:
\begin{align*}
    \begin{split}
        \mathbf{x}_t=\sqrt{1-\beta_t}\mathbf{x}_{t-1}+\sqrt{\beta_t}\epsilon_{t-1}\\
        \mathbf{x}_t=\sqrt{1-\beta_t}\Big(\sqrt{1-\beta_{t-1}}\mathbf{x}_{t-2}+\sqrt{\beta_{t-1}}\epsilon_{t-2}\Big)+\sqrt{\beta_t}\epsilon_{t-1}\\
        \mathbf{x}_t=\sqrt{(1-\beta_t)(1-\beta_{t-1})}\mathbf{x}_{t-2}+\sqrt{(1-\beta_t)\beta_{t-1}}\epsilon_{t-2}+\sqrt{\beta_T}\epsilon_{t-1}
    \end{split}
\end{align*}
服从相同均值的正态分布的两个随机向量之和所服从的分布,均值不变但方差是原随机变量方差之和。我们可以改写上式为
\begin{align*}
    \begin{split}
        =\sqrt{(1-\beta_t)(1-\beta_{t-1})}\mathbf{x}_{t-2}+\sqrt{(1-\beta_t)\beta_{t-1}+\beta_t}\epsilon\\
        =\sqrt{(1-\beta_t)(1-\beta_{t-1})}\mathbf{x}_{t-2}+\sqrt{1-(1-\beta_t)(1-\beta_{t-1})}\epsilon
    \end{split}
\end{align*}
将这样的过程一直推到$\mathbf{x}_0$,令$\alpha_t=1-\beta_t,\bar{\alpha}_t=\Pi_{i=1}^t\alpha_i$,则有
\begin{equation}
    \mathbf{x}_t=\sqrt{\bar{\alpha}_t}\mathbf{x}_0+\sqrt{1-\bar{\alpha}_t}\epsilon
\end{equation}
在DDPM的论文中,当$\beta_t$从$\beta_1=10^{-4}$到$\beta_T=0.02$线性增长,$\alpha_t$越来越小,$\bar{\alpha}_t$趋向于0的速度越来越快。
到最后的$T$时刻$\bar{\alpha}_T$几乎为0,这也使得$\mathbf{x}_T=\sqrt{\bar{\alpha}_T}\mathbf{x}_0+\sqrt{1-\bar{\alpha}_T}\epsilon$满足标准正态分布,
以符合扩撒模型加噪声的需求:从慢到快的改变原图像,让图像最终均值为$0$,方差为$\mathbf{I}$

\section{扩散模型加噪声逆操作均值和方差的推导}
在加噪声逆操作的数学推导部分,综合贝叶斯公式得到了
\begin{align*}
    q(x_{t-1}|x_t,x_0)=\mathcal{N}(x_{t-1};\tilde{\mu}_t,\tilde{\beta}_t\mathbf{I})
\end{align*}
